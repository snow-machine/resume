
%%% Local Variables: 
%%% mode: latex
%%% TeX-master: "resume"
%%% End: 

\section{個人簡歷}

我名叫邵勇,並非學軟件出身,酷愛程序設計,程序設計都是自學,從2002年從事軟件開發
以來,一直把軟件當成一門藝術來做,毎一個程序盡量做到精益求精。我的工作經歷可以分
為三個階段,2002至2005年,2005至2008年,2008至2012年,做過各方面軟件後,認識到軟
件是西方知識發展的極致,是人類設計出的最復雜的東西,但是缺少智慧,因為智慧是東方
文化發展出來的,相對知識,智慧對人更重要,目前想用知識(軟件)為智慧(文化)的啟迪和
傳播作一點事。

\subsection{2002年至2005年}

工作於國營長風機器廠,這是一家軍工企業,有4000名員工,我屬於研發部門。在這裏主要
作嵌入式開發,用c語言寫單片機程序,項目都不大,我一個人負責軟件開發,其中有兩個
項目值得一提:
\begin{itemize}
\item 便携式數字示波器:應用於軍隊,產品對顯示速度要求很高,用匯編和c混合編程,驗
  收時達到滿意的精度和速度
\item 風雲三號氣象衛星伺服軟件:與航天部510所合作,從模樣、初樣、正樣,用了三年時
  間,這個項目管理非常嚴格,對程序和文檔反復評審,力求做到產品的完美,自衛星升空
  至今運行正常
\end{itemize}

\subsection{2005年至2008年}

從長風辭職,就職於蘭州奧普信息技術有限公司,是甘肅省知名軟件企業,我擔任軟件部門
經理,做了很多項目,重要的有:
\begin{itemize}
\item 智能防溜鐵鞋:為蘭州鐵路局開發,產品分發c51開發的下位機,VB開發的上位機,
  之間通過無線方式通訊,系統架構和主要模塊由我負責,應用於蘭州至天水十余個火車站
\item 移動商務平臺:與甘肅移動公司合作,開發移動商務ADC服務平臺,使用J2EE架
  構,java語言,系統是網絡平臺,應用於甘肅各地市移動企業用戶。

  參與這個項目的有7名軟件人員,1名美工。平臺應用於甘肅各地市移動企業用戶,通過這
  個項目,我學到了如何管理技術團隊,與客戶溝通,應對各種需求的能加。更重要的是軟
  件設計水平有大幅度提高,用敏捷建模和迭代開發模式,使用開源架構,采用單元測試保
  證軟件質量,毎個迭代開發三周時間,有完整的設計、開發、測試流程。
\end{itemize}

\subsection{2008年至2012年}

蘭州奧普總部設到了北京,稱為北京奧普智信光科,研發搬到北京,位於豐臺科技園區。公
司業務轉向光纖光柵測量,在北京這段時間在各個方面都有很大的收獲。

從開發成果來說,研發成功了各種型號的光纖光柵測量軟件,有c51開發的下位機,csharp開
發的上位機,產品應用於30多個石油罐區,電力站。用c51開發中,使用了高煥堂先生介紹
的OOPC方式,使c有了面向對象能力,按照對象UML建模設計,使單片機程序在架構上有質的
飛躍,再加上對象的單元測試,單片機程序的質量也有大的提高。csharp開發中,大量使用
開源軟件,由於測量軟件對實時性要求高,其中數據庫采用開源的sqlite,響應速度快,不
需要安裝配置,為上位機成功使用提供了保證。

從技術上說,開始步入自由軟件的世界,自2009年,寫軟件使用emacs,寫文檔使用
\LaTeX{},繪圖使用asymptote,設計使用敏捷建模,寫代碼的同時編寫測試用例。

從精神世界說,了解了佛學,國學,認真閱讀和背誦心經,金剛經,道德經,知道了許多智
慧成長導師,逐漸不滿足在北京僅僅物質的生活。2009年,偶然的機會看到``老子止笑
譚'',嘆為難得的好書,但不知是何人所作,2011年,機緣到了,訪問到了``朱邦復工作
室''網站,看到``智慧之旅'',對朱先生用電腦為中國文化所做的事真心佩服,北京那個
公司只知道掙錢,感到越來越不舒服。從2012年開始,一直關註工作室,學用了倉頡輸入
法,聽了朱先生講的國學錄音,今年將手頭工作做一了結,於6月底辭職,又回到蘭州,過自
己想要的生活。

\newpage